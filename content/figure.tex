\section{Figures \& Codes}
\label{sec:fig}

\begin{figure}[thb]
\centering

\stackunder{\includegraphics[width=50mm,scale=0.5]{Figures/latex}}%
{\scriptsize%
            Source: \url{http://KeepCalmAndPosters.com/}}
\caption{Latex}
\label{fig:learnLatex}
\end{figure}
From the picture above \ref{fig:learnLatex} we have decided to learn latex and be COOL!

In order to add specific code in latex we have two ways of achieving it. The following code is written in latex but interpreted as python code,
\lstset{style=mystyle} 
Here is some Python code:
\begin{lstlisting}[language=Python, caption= Python Example Code]
num = float(input("Enter a number: "))
if num > 0:
   print("Positive number")
elif num == 0:
   print("Zero")
else:
   print("Negative number")
\end{lstlisting}
In this example the package xcolor used in the \textbf{main.tex} is imported and then the command \textbf{definecolor{}{}{}} is used to define new colours in rgb format that will later be used. 

There are essentially two commands that generate the style for this example:

\textbf{lstdefinestyle{mystyle}{...}}
Defines a new code listing style called "\textit{mystyle}". Inside the second pair of braces the options that define this style are passed.

The second method to display the python code without writing the code is by importing the python file. 

\lstinputlisting[language=Python]{PrimeNumber.py}

\clearpage

\lipsum[1-2]


%\begin{multicols}{2}
\begin{figure}[thb]
\centering

\stackunder{\includegraphics[width=0.4\textwidth]{Figures/latex}}%
{\scriptsize%
            Source: \url{http://KeepCalmAndPosters.com/}}
\caption{Latex}
\label{fig:learnLatexCompleteSize}
\end{figure}
%\end{multicols}

\lipsum[2-6]